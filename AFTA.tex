\section{Argument for the Affirmative}
\subsection{College is among the best time for Chinese students to learn music}
On the one hand, for Chinese middle school students, it becomes slightly impossible to learn something extra because their schedules are becoming tighter and tighter these days. The figure \ref{Timetable} shows a regular timetable for Chinese middle school students. 
\begin{figure}[h]
\centering
\includegraphics[scale=0.3]{timetable.png}
\label{Timetable}
\caption{A sample timetable for the Chinese high school}
\end{figure}
Under the pressure of entering a so-called good university, they have to work harder and harder. For primary school students, they are basically forced to learn something like the piano by their parents in order to gain extra grades in Gaokao. However, in 2008, the national educational administration had declared the cancellation of the extra grades policy, which discourage students from learning the piano or something like that. \par
On the other hand, as is showed in figure \ref{Comparison}: A Comparison Across Two Age Group, older adults seem less sensitive to emotional expressions in music than younger adults. Different age adults respond to music certain music to different extend. Besides, college students bear strong emotional inspirations and learn new things easily as their concentration is better than older adults. Combining college students’ better sensitivity and higher concentration level, it is more suitable to learn music in college than in the later stage.\par
\begin{figure}[h]
\centering
\includegraphics[scale=0.3]{introduction2.jpg}
\label{Comparison}
\caption{Comparsion of sensitivity to music between different adults age group}
\end{figure}
In conclusion, college is one of the best periods to learn music since it is a little bit impossible to learn in the later stage and worse to learn in the later stage.
\subsection{Music can inspire creativity in engineering}
The relationship between math, which is the basis of engineering, and music somehow exists. Some people like Emily Howard use math equations to organize her structures of compositions, as she says: ‘It’s the resulting collision and union of disparate ideas from diverse sources that excite me, and the subsequent translation of these hybrid ideas into sound is essentially the crux of my creative process.’ This, however, is the function of music on the music. Nevertheless, it at least proves that music does relate to math. As to creativity, “whether theological or not, art and science enact wisdom through creation”. Therefore, finding commonality will contributes to both of the realms.\par
However, “interdisciplinarity always raises problems of a deeply philosophical nature, but those problems often pale into insignificance in the face of more immediate concerns over disciplinary translation.” The inspiration provided by music for interdisciplinary needs at least professional concepts of music, like sonic vibrations, pitch, rhythm and volume to more advanced musical conceptions like harmony, metre and overarching tonal structures. Because in the realm of musicology, this is the basis for communicating the meaning of music, which is always conveying emotion. These kinds of professional knowledge can be taught with consistency and in a systematic way given professional instructors.\par
\subsection{They can relax after learning music}
Thomas Dekker said, “Sleep is that golden chain that ties health and our bodies together”. Sleep loss is a widespread problem with serious physical and economic consequences. Figure \ref{Figure1234}is the survey which basically shows the factors influencing college students’ sleeping quality, which include studying pressure, personal emotion, and busy work. figure \ref{Figure4321}shows that how music cures moods in several ways, which correspond well with the problems students are facing. As a result, music is a good choice for students to relax themselves.\par
\begin{figure} 
  \centering 
  \subfigure[Reasons for Loss of College Students' Sleeping]{ 
    %\label{fig:subfig:a} %% label for first subfigure 
    \includegraphics[scale=0.6]{reasons.png} 
    \label{Figure1234}
  } 
  \subfigure[Factors Music Helps Sleeping]{ 
    %\label{fig:subfig:b} %% label for second subfigure 
    \includegraphics[scale=0.3]{factors.png} 
    \label{Figure4321}
  } 
  \caption{Two Surveys} 
  %\label{fig:subfig} %% label for entire figure 
\end{figure}

However, how do music “course” matter? Students can just turn on the music on their phone and listen to some of them without taking any courses. Below are the genres involve in the survey above. Those in the top of the proportion are classical and rock. But if you turn on QQ music or NetEase cloud music and search for the most popular songs, most of them will be pop (as is shown in the figure). Therefore, we need professional instructions for college students to learn some classical music or something like that to gain real relax.\par
On top of that, a research on the rats shows regular exposure to music can also help reduce anxiety in juvenile period. But as college students have not become adults for a long time, and both human beings and rats are mammals (as is known, rats often serve as experimental material due to their similarity to human beings), this result can also be applied to college students, especially in China, where college students still highly depend on parents. A course is usually scheduled on a regular basis and can therefore contribute to reducing college students’ anxiety.\par

