\section{Introduction}
Many famous engineers are both engineers and musicians, or maybe they are both scientists and musicians. 
One of the alumni of SJTU, Xuesen Qian, is famous for his achievements on manufacturing missiles and something like that, but he is also good at music. At 98, Xuesen Qian said to people nationwide that one
 person should attempt to grasp not only science and technology, but also culture, art, and music. What he
  said implies the deep relationship between music and technology. In fact, Professor Qian’s wife, Ying 
  Jiang, is a well-known musician in China, even in the world. Einstein is also good at playing the
   violin. These masters are good examples of comprehensive interdisciplinarity.\par
From students’ perspective, as the widespread of electronic devices capable of playing music, college
 students are more accessible to music than they used to have. Some people, including students, report they
 can concentrate better with music while working. See figure \ref{Music and Concentration}.

\begin{figure}[h]
\centering
\includegraphics[scale=0.6]{introduction.png}
\label{Music and Concentration}
\caption{Music and Concentratiom}
\end{figure}

Music has already become a necessity of their life. But are they more satisfied with amateur state of listening to music, or acquiring some basic professional knowledge? There are various choices, they can learn by themselves, they can found clubs, or they take courses offered by the college. They are freer in the first two circumstances, but freedom does not equal quality. Which side of advantages is better for them?\par
On the other hand, from the data on USnews university rankings, many universities which is famous for their engineering schools are also famous for their art schools. This again shows a relationship between art and engineering. In other words, interdisciplinarity. As Walter Pater says ‘All art constantly aspires to the condition of music’, the relationship between art and engineering can be interpreted as the covering relationship between music and engineering. In a deep sense, can engineering schools improve themselves by offering music courses, or just encouraging students to take music course?\par
