\section{My Position}
Based on arguments for both sides, I think engineering students should be encouraged to take music as electives at different levels as they wish. These courses can be included in their GPA based on the student’s intention. Nevertheless, there should be related punishment if he or she get lower than C-. Of course, there are engineering students who do not like music or who do not need instruments to express their moods. But for those who really want to improve himself or herself in instruments or singing, they can benefit a lot from taking music courses. As they are in a course, they will have a scientific design which is made by an experienced and professional musical instructor. Additionally, they will have scheduled tests and examinations which can assist them check what they have achieved. All these make the music serious just as a subject should be like. Schools or institutions of engineering can also hire more honorable instructors than individuals can do. Besides, under the institution’s supervision, the instructor can also carry out his responsibilities in a better way. These ensure the cost of the curriculum worth the money. These courses can bring not only sense of achievement but the relax after monotonous work of chasing “due”s and so on. They may not be able to use math equations to compose a piece of music, but they unify elements in both realms which are not apparently related. The thinking of life and the meaning of it is far more important than engineering achievements because first of all, a person live in the world as a person, and in a boarder sense, as a piece of life. Utilizing music courses will eventually offers students not only professional knowledge of music but also an identity of thinker besides mere engineer. 

